%% Beamer document class
\documentclass[11pt]{beamer}
% Available font size options: 
% 8pt, 9pt, 10pt, 11pt (default), 12pt, 14pt, 17pt, 20pt

% Make printable version
%\usepackage{beamerarticle}

%% Language, fonts and encoding
\usepackage[english]{babel}
\usepackage[latin1]{inputenc}
\usepackage[T1]{fontenc}
%\usepackage[light,math]{iwona} % Math font w/o serifs
\usepackage[final]{microtype}
\usepackage{lmodern}

%% Links, colors, citations
%\usepackage{showkeys} % Show labels in pdf

%% Graphics
%\graphicspath{{./\jobname-graphics/}}
%\usepackage{subfig}
\usepackage[scriptsize]{caption} % Caption text style

%% Drawing
\usepackage{tikz}
\usetikzlibrary{shapes,arrows,3d,calc,decorations.pathmorphing}

%\usepackage{movie15}
\usepackage{multimedia}

%% Layout
%\usepackage{multicols} % n columns with \begin{multicols}{<n>} ... \end{multicols}
%\setlength{\columnsep}{5pc}

%% Beamer-specific layout and style
%\usepackage{beamerthemesplit} %Activate for custom appearance
%\usetheme{Berkeley}
%\usetheme{CambridgeUS}
\usetheme{Singapore}
\usecolortheme{seagull} % Gray color theme
%\usecolortheme{beetle} % Complete Color Theme
%%\usecolortheme{orchid} % Inner color theme
%%\usecolortheme{whale}  % Outer Color Theme
\xdefinecolor{AUblue}{rgb}{0.0,0.3,0.5}
%\setbeamercolor{sidebar}{bg=AUblue}
\setbeamercolor{frametitle}{fg=AUblue}
%\setbeamercolor{logo}{bg=AUblue!80!black}
%\setbeamercolor{normal text}{bg=white!20}
\setbeamercolor{section in toc}{fg=AUblue}
\setbeamercolor{title}{fg=red!50!black}

%\usefonttheme{serif} % Serif fonts

%% Mathematics, scientific and chemical notation
\usepackage{amssymb,amsmath,amsfonts}
%\usepackage{mathtools}
%\usepackage{booktabs} % \toprule, \midrule and \bottomrule in tabular
\usepackage[detect-all]{siunitx}
%\usepackage[version=3]{mhchem} % chemical notation

%% Code typesetting
\usepackage{listings}

% Do not show horizontal line above footnotes
\renewcommand\footnoterule{}


%% Citation commands

% \fcite{arg1}{arg2}
% Add a citation as a small footnote in the current slide. `arg1` will be 
% stylized as regular text, followed by `arg2` in emphasized style.
% Example: \fcite{Author year}{Journal}
\newcommand{\fcite}[2]{\let\thefootnote\relax\footnotetext{\footnotesize{#1 \emph{#2}}}}


%% Frame templates

% \titleframe{arg1}
% Add a plain frame with large centered text.
% Example: \titleframe{Conclusions}
\newcommand{\titleframe}[1]{\begin{frame}\begin{center}\huge{#1}\end{center}\end{frame}}

% \plainimageframe{arg1}{arg2}
% Add a plain frame with a single full-frame image `arg1`, scaled relative to 
% frame width multiplied by `arg2`.
% Example: \plainimageframe{image.png}{1.0}
\newcommand{\plainimageframe}[2]{\begin{frame}\begin{center}\centering\includegraphics[width=#2\textwidth]{#1}\end{center}\end{frame}}

% \plainimageframeheight{arg1}{arg2}
% Add a plain frame with a single full-frame image `arg1`, scaled relative to 
% frame height multiplied by `arg2`.
% Example: \plainimageframeheight{image.png}{1.0}
\newcommand{\plainimageframeheight}[2]{\begin{frame}\begin{center}\centering\includegraphics[height=#2\textheight]{#1}\end{center}\end{frame}}

% \imageframe{arg1}{arg2}{arg3}
% Add a frame with title `arg1`, containing a single full-frame image `arg2`, 
% scaled relative to frame width multiplied by `arg3`.
% Example: \imageframe{An example frame}{image.png}{1.0}
\newcommand{\imageframe}[3]{\begin{frame}{#1}\begin{center}\centering\includegraphics[width=#3\textwidth]{#2}\end{center}\end{frame}}

% \plainimageframeheight{arg1}{arg2}{arg3}
% Add a frame with title `arg1`, containing a single full-frame image `arg2`, 
% scaled relative to frame height multiplied by `arg3`.
% Example: \plainimageframeheight{An example frame}{image.png}{1.0}
\newcommand{\imageframeheight}[3]{\begin{frame}{#1}\begin{center}\centering\includegraphics[height=#3\textheight]{#2}\end{center}\end{frame}}

% \plainimageframe{arg1}{arg2}
% Add a frame with title `arg1`, containing a single full-frame image `arg2` 
% that is not scaled relative to the frame size.
% Example: \plainimageframe{An example frame}{image.png}
\newcommand{\imageframenoscale}[2]{\begin{frame}{#1}\begin{center}\centering\includegraphics{#2}\end{center}\end{frame}}

% \imageframecite{arg1}{arg2}{arg3}{arg4}{arg5}
% Add a frame with title `arg1`, containing a single image `arg2`, scaled 
% relative to frame height multiplied by `arg3`, including a citation to 
% publication authored by `arg4`, published in `arg5`.
% Example: \imageframecite{An example title}{image.pdf}{1.0}{Author year}{Journal}
\newcommand{\imageframecite}[5]{\begin{frame}{#1}\begin{center}\centering\includegraphics[width=#3\textwidth]{#2}\end{center}\fcite{#4}{#5}\end{frame}}


%%%% AUTHOR, TITLE, AFFILIATION
\title{\texttt{scibeamer}\\ a template for scientific presentations}
\author[A. Damsgaard]{Anders Damsgaard}
\institute[SIO]{Scripps Institution of Oceanography\\[2mm]
{\small \url{adamsgaard@ucsd.edu}}}
\date{\small Unmodified template, 2016-07-12}
%\logo{\includegraphics[width=30px]{logo1.pdf}}
%\logo{\includegraphics[width=30px]{logo2.pdf}}
 
\begin{document}

\begin{frame}[fragile]
  \begin{centering}
    \framesubtitle{Scripps Institution of Oceanography, UCSD}
    \titlepage{}
  \end{centering}
\end{frame}

% Optional slide with table of contents
%\begin{frame}[fragile]
    %\tableofcontents
%\end{frame}

\section{Introduction}

\begin{frame}{Usage}

    \texttt{scibeamer} is a flexible template containing functions for quickly 
    creating high-quality scientific presentations using \emph{beamer}.

    \vspace{1em}

    The functions in the \texttt{scibeamer} template are useful for including 
    images scaled to frame size with optional references to image source 
    publications.

\end{frame}

\begin{frame}{Building presentations}

    Add images to the folder containing \texttt{scibeamer.tex} and change the 
    \texttt{scibeamer.tex} to include the desired content, using the provided 
    functions or standard \emph{beamer} commands.

    \vspace{1em}

    The included \texttt{Makefile} allows quick output PDF generation by typing 
    \texttt{make} from the command line.

\end{frame}


\section{Examples}

\titleframe{Examples}

\plainimageframe{graphics/fig1.pdf}{1.0}

\plainimageframeheight{graphics/fig1.pdf}{0.5}

\imageframe{An example image frame using \texttt{\\imageframe}}
{graphics/fig1.pdf}{0.5}

\imageframenoscale{An example unscaled image frame using \texttt{\\imageframenoscale}}
{graphics/fig1.pdf}

\imageframecite{An example image frame using \texttt{\\imageframecite}}
{graphics/fig1.pdf}{0.8}%
{Damsgaard et al. 2013}{J. Geophys. Res.-Earth}


\section{Conclusions}
\begin{frame}{Conclusions}
    \begin{itemize}

        \item \texttt{scibeamer} adds to \emph{beamer} functionality by 
            providing functions for quickly creating image-based slides.\\[3mm]

        \item Output files can be built as \texttt{pdf} (default), \texttt{dvi}, 
            or \texttt{eps} using the provided \texttt{Makefile}.\\[3mm]

        \item \texttt{scibeamer} is available at 
            \url{https://github.com/anders-dc/scibeamer}.

    \end{itemize}
\end{frame}


%%%% APPENDIX
%\titleframe{Appendix}

%\begin{frame}
%    Optionally include appendix content here.
%\end{frame}

\end{document}
